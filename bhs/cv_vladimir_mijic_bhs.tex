%----------------------------------------------------------------------------------------
%	PACKAGES AND OTHER DOCUMENT CONFIGURATIONS
%----------------------------------------------------------------------------------------
\documentclass[9pt]{developercv} % Default font size, values from 8-12pt are recommended
%----------------------------------------------------------------------------------------

\begin{document}

%----------------------------------------------------------------------------------------
%	TITLE AND CONTACT INFORMATION
%----------------------------------------------------------------------------------------

\begin{minipage}[t]{0.45\textwidth}
	\vspace{-\baselineskip}
	{\Huge\textbf{\MakeUppercase{Vladimir Mijić}}}
	
	\vspace{6pt}
	
	{\huge AI/ML inženjer}
\end{minipage}

\vspace{1cm}

\begin{minipage}[t]{0.3\textwidth}
    \vspace{-\baselineskip}
    
	{\faMapMarker \space}{Banja Luka, BiH}\\\\
\end{minipage}
\hfill
\begin{minipage}[t]{0.3\textwidth}
	\vspace{-\baselineskip}
	
	{\faEnvelopeO \space}{\href{mailto:vladocodes@gmail.com}{vladocodes@gmail.com}}\\
\end{minipage}
\hfill
\begin{minipage}[t]{0.3\textwidth}
	\vspace{-\baselineskip}
	
    {\faGithub \space}{\href{https://github.com/neuralmaticv}{GitHub}}\\
    {\faLinkedin \space}{\href{https://www.linkedin.com/in/vladimir-mijic}{LinkedIn}}
\end{minipage}

\vspace{0.1cm}

%----------------------------------------------------------------------------------------
%	EXPERIENCE
%----------------------------------------------------------------------------------------

\colorbox{black}{{\textcolor{white}{\textbf{\MakeUppercase{Iskustvo}}}}}
\par\noindent\rule{\textwidth}{2px}\\

    {\textbf{AI/ML inženjer u kompaniji LANACO}}\\
    {\faCalendarCheckO \space}{od oktobra 2024.}
    \begin{itemize}
        \item Istraživanje i razvoj u oblasti obrade prirodnog jezika, s fokusom na primjenu vještačke inteligencije.
        \item Razvio sistem za detekciju zauzeća parking mjesta zasnovan na računarskom vidu, kao dio LANACO smart city platforme.
        \item Takođe radio na razvoju sistema za obradu podataka sa senzora u realnom vremenu koristeći edge computing i LoRaWAN.
        \item Mentorisao grupu od 10 studenata tokom AI/ML prakse na temu streaming podataka u realnom vremenu. Držao predavanja, organizovao radionice i vodio razvoj projekata.
        \item \textbf{Tehnologije}: Python, PyTorch, OpenCV, Hugging Face stack, MQTT, GStreamer, NVIDIA Jetson platforma (JetPack, TensorRT, CUDA, DeepStream), Docker, Git\\
    \end{itemize}

    {\textbf{Stipendista – Istraživač u oblasti AI/ML u kompaniji LANACO}}\\
    {\faCalendarCheckO \space}{oktobar 2022. -- oktobar 2024.}
    \begin{itemize}
        \item Učestvovao u razvoju rekomender sistema zasnovanog na pretrazi po vizuelnoj sličnosti.
        \item Razvio PoC sistem za automatsko prepoznavanje registarskih tablica korišćenjem dubokog učenja za detekciju i prepoznavanje karaktera.
        \item Ko-mentor grupe od 15 studenata tokom AI/ML prakse, održao radionicu o retrieval-augmented generation koristeći open-source LLM modele.
        \item \textbf{Tehnologije}: Python, PyTorch, TensorFlow, OpenCV, Hugging Face Transformers, Linux, Docker, Git\\
    \end{itemize}

    {\textbf{Praksa iz embedded softvera u kompaniji SYRMIA (danas dio HTEC grupe)}}\\
    {\faCalendarCheckO \space}{april 2024. \hfill \textit{(kratkoročna studentska praksa)}}
    \begin{itemize}
        \item Stečeno praktično iskustvo u razvoju embedded softvera za autoindustriju.
        \item Učestvovao u verifikaciji bezbjednosti sistema i automatizaciji build procesa koristeći CI alate i unit testove.
        \item Korišćenje Elastic Stack-a za monitoring sistema, agregaciju i prikaz podataka.
        \item \textbf{Tehnologije}: C, Elasticsearch, Logstash, Kibana, Git\\
    \end{itemize}

\vspace{0.5cm}

%----------------------------------------------------------------------------------------
%	EDUCATION
%----------------------------------------------------------------------------------------

\colorbox{black}{{\textcolor{white}{\textbf{\MakeUppercase{Obrazovanje}}}}}
\par\noindent\rule{\textwidth}{2px}\\

    {\huge \textbf{Diplomske studije iz računarstva} \smallskip}\\
    \textbf{Prirodno-matematički fakultet, Univerzitet u Banjoj Luci}\\
    {\faCalendarCheckO \space}oktobar 2020 -- danas \hfill \textit{očekivano diplomiranje: decembar 2025.}

    \vspace{0.3cm}

	{\huge Tehničar telekomunikacija \smallskip}\\
    {\textbf{Elektrotehnička škola "Nikola Tesla", Banja Luka}}\\
    {\faCalendarCheckO \space}{septembar 2016 -- maj 2020}\\

\vspace{2cm}

%----------------------------------------------------------------------------------------
%	ADDITIONAL
%----------------------------------------------------------------------------------------

\colorbox{black}{{\textcolor{white}{\textbf{\MakeUppercase{Dodatno profesionalno usavršavanje i postignuća}}}}}
\par\noindent\rule{\textwidth}{2px}
\begin{itemize}
    \item Alumni i volonter Fondacije budućnosti u BiH (BHFF)
    \item Volonter u mreži CODE4SEE (od 2023.)
    \item Treće mjesto – Youth Innovation Award, Sarajevo, decembar 2023.
    \item Volonter – Data Science Conference Europe, Beograd, novembar 2023.
    \item Treće mjesto – Takmičenje iz mašinskog učenja na "barKod" hakatonu, Novi Sad, februar 2023.
    \item Pridruženi član Asocijacije za digitalnu transformaciju u BiH (2021–2022); pokrenuo inicijativu Open Data BiH.
\end{itemize}
\end{document}
